%%%%%%%%%%%%%%%%%%%%%%%%%%%%%%%%%%%%%%%%%
% Programming/Coding Assignment
% LaTeX Template
%
% This template has been downloaded from:
% http://www.latextemplates.com
%
% Original author:
% Ted Pavlic (http://www.tedpavlic.com)
%
% Note:
% The \lipsum[#] commands throughout this template generate dummy text
% to fill the template out. These commands should all be removed when 
% writing assignment content.
%
% This template uses a Perl script as an example snippet of code, most other
% languages are also usable. Configure them in the "CODE INCLUSION 
% CONFIGURATION" section.
%
%%%%%%%%%%%%%%%%%%%%%%%%%%%%%%%%%%%%%%%%%

%----------------------------------------------------------------------------------------
%	PACKAGES AND OTHER DOCUMENT CONFIGURATIONS
%----------------------------------------------------------------------------------------

%\documentclass{article}
%
%\usepackage{fancyhdr} % Required for custom headers
%\usepackage{lastpage} % Required to determine the last page for the footer
%\usepackage{extramarks} % Required for headers and footers
%\usepackage[usenames,dvipsnames]{color} % Required for custom colors
%\usepackage{graphicx} % Required to insert images
%\usepackage{courier} % Required for the courier font
%\usepackage{lipsum} % Used for inserting dummy 'Lorem ipsum' text into the template
%\usepackage{amsmath,amsfonts,amsthm} %Math Packages
%\usepackage{mathabx}
%\usepackage{mathtools}
%\usepackage{amssymb}
%\usepackage{hyperref}
%\usepackage{mdframed}
%\usepackage{array}
%%amsmath customizations
%\makeatletter
%\renewcommand*\env@matrix[1][*\c@MaxMatrixCols c]{%
%  \hskip -\arraycolsep
%  \let\@ifnextchar\new@ifnextchar
%  \array{#1}}
%\makeatother
%
%% So I can find things I left missing
%\newcommand{\fixme}{{\Huge \color{red} FIXME }}
%
%\usepackage{tikz}
%\usepackage{pgfplots}
%\usetikzlibrary{pgfplots.polar}
%% Margins
%\topmargin=-0.45in
%\evensidemargin=0in
%\oddsidemargin=0in
%\textwidth=6.5in
%\textheight=9.0in
%\headsep=0.25in
%
%\linespread{1.1} % Line spacing
%
%% Set up the header and footer
%\pagestyle{fancy}
%\lhead{\docAuthorName} % Top left header
%\chead{\docClass\ (\docAdmin\ \docTime): \docTitle} % Top center head
%\rhead{\firstxmark} % Top right header
%\lfoot{\lastxmark} % Bottom left footer
%\cfoot{} % Bottom center footer
%\rfoot{Page\ \thepage\ of\ \protect\pageref{LastPage}} % Bottom right footer
%\renewcommand\headrulewidth{0.4pt} % Size of the header rule
%\renewcommand\footrulewidth{0.4pt} % Size of the footer rule
%
%\setlength\parindent{0pt} % Removes all indentation from paragraphs
%
%%----------------------------------------------------------------------------------------
%%	CODE INCLUSION CONFIGURATION
%%----------------------------------------------------------------------------------------
%
%\definecolor{dkred}{rgb}{0.85 0 0}% dark red 
%\definecolor{MyDarkGreen}{rgb}{0.0,0.4,0.0} % This is the color used for comments
%\definecolor{light-gray}{gray}{0.95} %Light gray for zebra background
%
%%----------------------------------------------------------------------------------------
%%	DOCUMENT STRUCTURE COMMANDS
%%	Skip this unless you know what you're doing
%%----------------------------------------------------------------------------------------
%
%% Header and footer for when a page split occurs within a policy environment
%\newcommand{\enterPolicyHeader}[1]{
%\nobreak\extramarks{#1}{#1 continued on next page\ldots}\nobreak
%\nobreak\extramarks{#1 (continued)}{#1 continued on next page\ldots}\nobreak
%}
%
%% Header and footer for when a page split occurs between policy environments
%\newcommand{\exitPolicyHeader}[1]{
%\nobreak\extramarks{#1 (continued)}{#1 continued on next page\ldots}\nobreak
%\nobreak\extramarks{#1}{}\nobreak
%}
%
%\setcounter{secnumdepth}{0} % Removes default section numbers
%\newcounter{policyCounter} % Creates a counter to keep track of the number of policies
%\setcounter{policyCounter}{0}
%
%\newcommand{\policyName}{}
%\newenvironment{policy}[1][\arabic{policyCounter}]{ % Makes a new environment called policy which takes 1 argument (custom name) but the default is "Policy #"
%\stepcounter{policyCounter} % Increase counter for number of policies
%\renewcommand{\policyName}{#1} % Assign \policyName the name of the policy
%\section{\policyName} % Make a section in the document with the custom policy count
%\enterPolicyHeader{\policyName} % Header and footer within the environment
%}{
%\exitPolicyHeader{\policyName} % Header and footer after the environment
%}
%
%\newcommand{\policyResponse}[1]{ % Defines the policy response command with the content as the only argument
%\noindent\framebox[\columnwidth][c]{\begin{minipage}{0.98\columnwidth}#1\end{minipage}} % Makes the box around the policy response and puts the content inside
%}
%
%\newcommand{\policySectionName}{}
%\newenvironment{policySection}[1]{ % New environment for sections within policies, takes 1 argument - the name of the section
%\renewcommand{\policySectionName}{#1} % Assign \policySectionName to the name of the section from the environment argument
%\subsection{\policySectionName} % Make a subsection with the custom name of the subsection
%\enterPolicyHeader{\policyName\ [\policySectionName]} % Header and footer within the environment
%}{
%\enterPolicyHeader{\policyName} % Header and footer after the environment
%}
%
%%----------------------------------------------------------------------------------------
%%	NAME AND CLASS SECTION
%%----------------------------------------------------------------------------------------
%
%\newcommand{\docTitle}{Cloud Services} % Assignment title
%\newcommand{\draftDate}{June 09, 2015} % Due date
%\newcommand{\docClass}{Policies} % Course/class
%\newcommand{\docTime}{2015} % Class/lecture time
%\newcommand{\docAdmin}{Official Policy} % Teacher/lecturer
%\newcommand{\docAuthorName}{UNM Informatics Lab} % Your name
%
%%----------------------------------------------------------------------------------------
%%	TITLE PAGE
%%----------------------------------------------------------------------------------------
%
%\title{
%\vspace{2in}
%\textmd{\textbf{\docClass:\ \docTitle}}\\
%\normalsize\vspace{0.1in}\small{Drafted on\ on\ \draftDate}\\
%\vspace{0.1in}\large{\textit{\docAdmin\ \docTime}}
%\vspace{3in}
%}
%
%\author{\textbf{\docAuthorName}}
%\date{\draftDate} % Insert date here if you want it to appear below your name
%
%%----------------------------------------------------------------------------------------
%
\documentclass{article}
\usepackage{policy}
\begin{document}

\maketitle

%----------------------------------------------------------------------------------------
%	TABLE OF CONTENTS
%----------------------------------------------------------------------------------------

%\setcounter{tocdepth}{1} % Uncomment this line if you don't want subsections listed in the ToC

\newpage
\tableofcontents
\newpage

% To have just one policy per page, simply put a \clearpage after each policy

\begin{policy}[AC-1]
  \policyName{ Access Control Policy and Procedures}
  \policySection{Purpose}
  The purpose of this section is to document and communicate the process of
  Access Control and issuance of access to any cloud-based system covered by
  this document.
  \policySection{Scope}
  This document applies to all systems managed by the UNM Inforamtics lab that
  may contain any sensitive information. Sensitive information is anything that
  is covered by UNM Policy 2520 Section 5.2 \cite{UNM2520}.
\end{policy}
%----------------------------------------------------------------------------------------

\begin{policy}[AC-2]
  \policyName{ Account Management}\label{AC2}
  \policySection{(1)}
  \policySectionName{ Automated System Account Management}
  This system employs various techniques in order to automatically manage
  systems accounts, including provisioning and deprovisioning accounts, and
  verifying identities. All authentication will be through a public key meeting
  at least \fixme standard, or a password meeting \fixme standard.

  An administrator will be notified by e-mail automatically if any changes are
  made to any user's permissions on the system. This includes, but is not
  limited to, the following operations:
  \begin{itemize}
    \item A user account is provisioned on the system.
    \item A user is terminated or transferred.
    \item A user account is deprovisioned on the system.
    \item A user account is undergoing atypical usage.
    \item A system previously containing no sensitive information will have
      access to sensitive information.
  \end{itemize}
  \policySection{(2)}
  \policySectionName{ Removal of Temporary / Emergency Accounts}
  Temporary and emergency accounts are accounts intended for short-term use.
  They will be removed if any of the following conditions are met:
  \begin{itemize}
    \item When the account is no longer required.
    \item When individuals with access to the account are transferred or
      terminated.
    \item When an agreed termination date and time for the account has been
      reached.
  \end{itemize}
  \policySection{(3)}
  \policySectionName{ Disable Inactive Accounts}
  Accounts will be automatically disable if inactive for more than 90 days.
  \policySection{(4)}
  \policySectionName{ Automated Audit Actions}
  Any action related to the creation, change, or removal of a user from the
  system will be logged. In addition to the actions listed in
  \hyperref[AC2]{policy AC-2}, the following actions will be audited and
  logged:
  \begin{itemize}
    \item User account enabling
    \item User account disabling
    \item User account modification
    \item Account inactivity
  \end{itemize}
\end{policy}

\begin{policy}[AC-3]
  \policyName{ Access Enforcement}
  Access enforcement will be performed according to an access control matrix
  like the example \hyperref[control:matrix]{here}. All identity verification will be performed either using
  public-key cryptography meeting at least the standards here \fixme, or a
  password meeting the standards at \fixme.

  \begin{table}[h]
    \label{control:matrix}
    \begin{tabular}{ccccccc}
      & \textbf{Asset 1} & \textbf{Asset 2} & \ldots & \textbf{Asset N} &
      \textbf{File} & \textbf{Device} \\ \hline \hline
      \textbf{Role 1} & \{\emph{read,write,execute,own}\} &
      \{\emph{read,write,execute,own}\} & & \ldots & \{\emph{read,write\}} &
      \{\emph{read,write}\} \\ \hline
      \textbf{Role 2} & \{\emph{read,write,execute,own}\} &
      \{\emph{read,write,execute,own}\} & & \ldots & \{\emph{read,write\}} &
      \{\emph{read,write}\} \\ \hline
      \ldots \\ \hline
      \textbf{Role N} & \{\emph{read,write,execute,own}\} &
      \{\emph{read,write,execute,own}\} & & \ldots & \{\emph{read,write\}} &
      \{\emph{read,write}\} \\ \hline
    \end{tabular}
    \caption{Sample Access Control Matrix}
  \end{table}
\end{policy}

\begin{policy}[AC-4]
  \policyName{ Information Flow Enforcement}
  Information flow enforcement will be conducted by attaching metadata to
  sensitive data, and establishing secure information domains. Information with
  metadata attached will always be encrypted in transit, and will only use
  approved information transit mechanisms, such as \texttt{sftp}.
\end{policy}

\begin{policy}[AC-5]
  \policyName{ Separation of Duties}
  Separation of duties will be implemented primarily to prevent the following
  scenarios:
  \begin{itemize}
    \item Destruction of important data by an authenticated individual.
    \item Exfiltration of sensitive data by an authenticated individual.
  \end{itemize}
  In order to reduce these risks, the following controls will be implemented:
  \begin{itemize}
    \item Servers containing sensitive data may only be accessed from a
      designated client machine on campus.
    \item Servers running important services may only be accessed from a
      designaed client machine on campus.
    \item No unnecessary tools will be installed on production servers. This
      includes but is not limited to the following:
      \begin{itemize}
        \item Development Tools
        \item Web Access Tools
        \item Debugging Tools
        \item Profiling Tools
      \end{itemize}
    \item Backups will be automated and cannot be triggered manually without
      approval from an assigned administrator and a written log detailing the
      incident requiring a manual backup.
  \end{itemize}
\end{policy}

\begin{policy}[AC-6]
\end{policy}

\begin{policy}[References]
  \bibliographystyle{plain}
  \bibliography{bib}
\end{policy}

%----------------------------------------------------------------------------------------


\end{document}
